%% For double-blind review submission
\documentclass[sigplan,10pt]{acmart}\settopmatter{printfolios=true}
%% For single-blind review submission
%\documentclass[sigplan,10pt,review]{acmart}\settopmatter{printfolios=true}
%% For final camera-ready submission
%\documentclass[sigplan,10pt]{acmart}\settopmatter{}

%% Note: Authors migrating a paper from traditional SIGPLAN
%% proceedings format to PACMPL format should change 'sigplan' to
%% 'acmsmall'.

\usepackage[T1]{fontenc}
\usepackage{lmodern}
%% Some recommended packages.
\usepackage{booktabs}   %% For formal tables:
                        %% http://ctan.org/pkg/booktabs
\usepackage{subcaption} %% For complex figures with subfigures/subcaptions
                        %% http://ctan.org/pkg/subcaption
\usepackage{listings}
\usepackage{float}
\usepackage{dblfloatfix}
\usepackage{placeins}

\lstset{
frame = single,
basicstyle=\small
}

%% Copyright information
%% Supplied to authors (based on authors' rights management selection;
%% see authors.acm.org) by publisher for camera-ready submission
\setcopyright{none}             %% For review submission
%\setcopyright{acmcopyright}
%\setcopyright{acmlicensed}
%\setcopyright{rightsretained}
%\copyrightyear{2017}           %% If different from \acmYear


%% Bibliography style
\bibliographystyle{ACM-Reference-Format}
%% Citation style
%% Note: author/year citations are required for papers published as an
%% issue of PACMPL.
%\citestyle{acmauthoryear}  %% For author/year citations
%\citestyle{acmnumeric}     %% For numeric citations
%\setcitestyle{nosort}      %% With 'acmnumeric', to disable automatic
                            %% sorting of references within a single citation;
                            %% e.g., \cite{Smith99,Carpenter05,Baker12}
                            %% rendered as [14,5,2] rather than [2,5,14].
%\setcitesyle{nocompress}   %% With 'acmnumeric', to disable automatic
                            %% compression of sequential references within a
                            %% single citation;
                            %% e.g., \cite{Baker12,Baker14,Baker16}
                            %% rendered as [2,3,4] rather than [2-4].



\begin{document}

%% Title information
\title[Metaborg Typescript]{Implementing Typescript in Metaborg Spoofax}         %% [Short Title] is optional;
                                        %% when present, will be used in
                                        %% header instead of Full Title.
%% Author information
%% Contents and number of authors suppressed with 'anonymous'.
%% Each author should be introduced by \author, followed by
%% \authornote (optional), \orcid (optional), \affiliation, and
%% \email.
%% An author may have multiple affiliations and/or emails; repeat the
%% appropriate command.
%% Many elements are not rendered, but should be provided for metadata
%% extraction tools.

%% Author with single affiliation.
\author{Tim van der Lippe}

%% Author with two affiliations and emails.
\author{Thomas Smith}

%% Abstract
%% Note: \begin{abstract}...\end{abstract} environment must come
%% before \maketitle command
\begin{abstract}
TypeScript is a structurally typed superset of JavaScript that aims to ease the introduction of static types
to large JavaScript code bases. In this paper, we investigate how the static semantics of TypeScript can be
modeled using NaBL2. We find that while NaBL2 misses some features needed to support structural types, 
these features can be implemented in terms of the existing abstractions that NaBL2 is built on.
We show that Scope Graphs can be used to model structural types and propose a minimal set of extensions
to NaBL2 that enable language designers to define custom subtype relations. We also show that these extensions
are enough to implement most major features of the TypeScript language.
\end{abstract}

%% Keywords
%% comma separated list
\keywords{Metaborg, Spoofax, Typescript, NaBL2}  %% \keywords is optional


%% \maketitle
%% Note: \maketitle command must come after title commands, author
%% commands, abstract environment, Computing Classification System
%% environment and commands, and keywords command.
\maketitle


\section{Introduction}

TypeScript is a statically typed programming language that is a superset JavaScript (ECMAScript 2015).
The aim of TypeScript is to provide large JavaScript projects with a way to integrate a static 
type system into their code in a non-intrusive way, 
while keeping the dynamic semantics equivalent to JavaScipt.
To achieve this, TypeScript compiles to human-readable JavaScript with as little transformations as possible.
Any value that has type annotations is checked by the type checker.
In addition, some type inference is done to provide static checks even for existing JavaScript without annotations.
Only when no type information can be inferred will the compiler fall back to default JavaScipt semantics.
The system of structural types enables TypeScript to be non-intrusive when used with existing JavaScript projects.
It allows programmers to gain some of benefits of static analysis while still 
being able to use existing JavaScript APIs, libraries and coding conventions.

In this report we explore how structural type systems could be implemented in terms of Scope 
Graphs with NaBL2\citep{Antwerpen:2016:CLS:2847538.2847543}.
We find that while NaBL2 already works very well for nominally typed languages with classes and inheritance,
structural types are not yet fully supported. 

Our main contribution is the proposal of a minimal set of extensions to NaBL2 that 
give language designers the possibility to define their own subtype relations. 
We show that the fundamental model of Scope Graphs is viable even for structural types, 
and that making constraint generation and resolution interleaved
allows language designers to model more complex and diverse type systems with NaBL2.

An introduction to structural type systems is given in Section \ref{sec:structural-typing}.
Then, the most significant features of the TypeScript language are discussed in Section \ref{sec:structural-typing-typescript}.
After the introduction of some of TypeScripts features, we move on to discuss, in Section \ref{sec:syntax}, how SDF3 could be used to 
easilly mirror the ECMAScript 2015 specification with the additional syntactical constructs of TypeScript.
Finally, Section \ref{sec:type-checking} discusses how structural typing can be supported in NaBL2 
as well as possible implementations of some TypeScript features in NaBL2.

\section{Structural Typing}
This section gives a high level overview of how structural type systems work.
Structural subtyping stands opposed to nominal subtyping. 
A majority of the most used programming languages employ a nominal type system.
In such languages, subtype relationships are explicitly declared.
The programmer has to relate two types to each other by their name,
after which the compiler will check whether the structures refered to by these names are compatible.
This process has to be done only once for each declared subtype relation.
As soon as the compiler has established that a type A is a subtype of B,
the next time that a check for this relationship has to be performed, 
a simple table lookup for the two names is sufficient.
\bigskip
With structural subtyping on the other hand, the relationships between types are implicit.
The programmer does not have to specify any relationship by hand,
every time the compiler has to check compatibility between two types, a structural comparison of the types is done.
One major consequence of this approach is that names for types become purely cosmetic.
The programmer can still assign names to types, and refer to those types by their names,
but checking whether one type is a subtype of another is completely dependent on their structure.
Generally this means that a type A is a subtype of B when the structure of A has at least as much information as the strucrure of B.
This means that a value of type A can be assigned to B by simply removing all information present in A that is not required by B.
For most languages that implement structural subtyping, the information that types carry consists of names paired with another type.
For example, in TypeScript we may write:

\begin{lstlisting}
type A = { first: string, second: string }
\end{lstlisting}

This defines the type A to be an object that contains two named fields of type 'string'.
Values of type A can be passed to anywhere a subset of \texttt{\{ first: string, second: string \}} is required.
So all the following statements are valid because the types of $x$, $y$ and $z$ are structurally subsets of $A$:

\begin{lstlisting}
const a: A =  { first: 'x', second: 'y' }

const x: { first: string } = a
const y: { second: string } = a
const z: {} = a
\end{lstlisting}

The nature of such type systems makes implementing a type checker slightly more difficult than with a nominal type system.
To demonstrate this, we consider recursive types:
\begin{lstlisting}
type Nil = {}
type Cons = { head: number, tail: NumList }
type NumList = Cons | Nil
\end{lstlisting}

The \texttt{|} operator describes that $NumList$ can either be $Cons$ nil $Nil$.
Note that $Cons$ and $NumList$ are mutually recursive.
To check whether we can assign a $NumList$ value to another value, 
we need to check if their unfoldings are equal.
Since unfolding a recursive type yields an infinite tree, 
the type checker has to make sure that this process terminates regardless of the possibly infinite type.
This approach to type checking is called \textit{equi-recursive} because a recursive type should
be a subtype of its one step unfolding.

\section{Structural Typing in Typescript}
\label{sec:structural-typing-typescript}
TypeScript was built with structural typing as it's backbone, 
nearly any feature in the language can be described in terms of structual types.
The reason for this being that the goal of the language is to provide a type system
on top of the dynamically typed JavaScript language while not introducing any 
abstractions that have a conceptually high distance to the semantics of JavaScript.
The result of this effort means that compiled TypeScript code should be suitable 
for human consumption. Structural typing provides an excellent bridge from the 
dynamically typed JavaScript world to the statically typed world of TypeScript.
In this section we will introduce the most important features of TypeScript and
how they relate to structural types.
\\
\\
Structural types manifest themselves in TypeScript as \textit{Record Types}.
In the examples above, we have already encountered some record types, a set of name-type pairs.
Built directly on top of record types are some of the most important aspects of the language:
\begin{itemize}
\item classes
\item interfaces
\item union types
\item intersection types
\end{itemize}

Other notable features include: first class and higher order functions, parametric and ad-hoc polymorphism 
(generics) and several types for special values like \texttt{null} and \texttt{undefined}. It is 
worth noting that the subtype relation within TypeScript does not form a proper latice due to 
the \texttt{any} type. Namely, a value of type \texttt{any} is a subtype of every other type, 
while every type is simultaniously a subtype of any.
\\
\\
- describe classes / interfaces in terms of sugar for records
- describe union / intersection in terms of operations on records

\section{Syntax definition}
\label{sec:syntax}
The syntax definition of Typescript is defined in the TypeScript Language Specification.
Even though this specification is not up-to-date for TypeScript 2, the syntax has not significantly changed between TypeScript 1 and 2.
The specification builds upon the ECMAScript 2015 Language Specification.
While it is defined as a superset with additions to the syntax definitions, several productions in the TypeScript grammar modify or replace the productions with the same name in the ECMAScript definition.

The translation of the syntax definition into the SDF3\citep{Vollebregt:2012:DST:2427048.2427056} productions starts with the ECMAScript productions and is then updated with the TypeScript productions.
All SDF3 production symbol names are taken from the language specification(s), while the constructor labels are either a copy of the symbol name or a self-invented name.
An example of a translation of \textit{IfStatement} is shown in Figure \ref{fig:if-statement}.
For multi-line productions, whitespace and indentation is inserted based on personal preferences of the authors.
The SDF3 templates adhere to the indentation to achieve formatted code completion in the editor, which the language specification does not.
The ECMAScript language specification also includes parameters, denoted by a suffix in subscript, which are disregarded in the SDF3 production notation.

\begin{figure*}
  \begin{lstlisting}[caption=Definition of \textit{IfStatement} in the ECMAScript language specification,mathescape]
IfStatement$\textsubscript{[Yield, Return]}$ : See 13.6
  if ( Expression$\textsubscript{[In, ?Yield]}$ ) Statement$\textsubscript{[?Yield, ?Return]}$ else Statement$\textsubscript{[?Yield, ?Return]}$
  if ( Expression$\textsubscript{[In, ?Yield]}$ ) Statement$\textsubscript{[?Yield, ?Return]}$
  \end{lstlisting}
  \begin{lstlisting}[caption=Definition of \textit{IfStatement} in SDF3 production notation]
IfStatement.IfElse = <
  if (<PrimaryExpression>) <Statement>
  else <Statement>
>
IfStatement.If = <if (<PrimaryExpression>) <Statement>>
  \end{lstlisting}
  \caption{The translation of \textit{IfStatement} from the ECMAScript language specification to SDF3 production notation.}
  \label{fig:if-statement}
\end{figure*}

Some productions in the ECMAScript only directly inject other productions.
For these productions, the SDF3 notation does not define a constructor label, unless the label is required in the type-checking phase to prevent ambiguation on the constructor labels in the constraint generation phase.
An example of a production with only directly injected other productions is shown in Figure \ref{fig:direct-production-spec} and \ref{fig:direct-production-sdf}.
In the example, the extra constructor label \textit{ModuleStatement} is defined, to prevent disambiguation in programs and modules.

\begin{figure}[H]
\begin{lstlisting}
ModuleItem : See 15.2
  ImportDeclaration
  ExportDeclaration
  StatementListItem
\end{lstlisting}
\caption{Definition of \textit{ModuleItem} in the ECMAScript language specification}
\label{fig:direct-production-spec}
\end{figure}

\begin{figure}[H]
\begin{lstlisting}
ModuleItem = ImportDeclaration
ModuleItem = ExportDeclaration
ModuleItem.ModuleStatement = 
  StatementListItem
\end{lstlisting}
\caption{Definition of \textit{ModuleItem} in SDF3}
\label{fig:direct-production-sdf}
\end{figure}


% \begin{figure*}
%   \begin{minipage}[t]{.8\columnwidth}
%     \begin{lstlisting}[caption=Definition of \textit{ModuleItem} in the ECMAScript language specification]
% ModuleItem : See 15.2
%   ImportDeclaration
%   ExportDeclaration
%   StatementListItem
%     \end{lstlisting}
%   \end{minipage}
%   \hfill
%   \begin{minipage}[t]{1.1\columnwidth}
%     \begin{lstlisting}[caption=Definition of \textit{ModuleItem} in SDF3 production notation]
% ModuleItem = ImportDeclaration
% ModuleItem = ExportDeclaration
% ModuleItem.ModuleStatement = StatementListItem
%     \end{lstlisting}
%   \end{minipage}
%   \caption{The translation of \textit{ModuleItem} from the ECMAScript language specification to SDF3 production notation.}
%   \label{fig:direct-production}
% \end{figure*}

Lastly, Figure \ref{fig:override-syntax} shows a production that is modified by TypeScript and thus overrides the original definition from the ECMAScript language specification.
Note that the second production as defined in the TypeScript language definition has not been implemented in SDF3 production notation at the time of writing.

\begin{figure*}
  \begin{lstlisting}[caption=Definition of \textit{FunctionDeclaration} in the ECMAScript language specification,mathescape]
FunctionDeclaration$\textsubscript{[Yield, Default]}$ : See 14.1
  function BindingIdentifier$\textsubscript{[?Yield]}$ ( FormalParameters ) { FunctionBody }
  [+Default] function ( FormalParameters ) { FunctionBody }
  \end{lstlisting}
  \begin{lstlisting}[caption=Definition of \textit{FunctionDeclaration} in the TypeScript language specification,mathescape]
FunctionDeclaration: ( Modified )
  function BindingIdentifier$\textsubscript{opt}$ CallSignature { FunctionBody }
  function BindingIdentifier$\textsubscript{opt}$ CallSignature ;
  \end{lstlisting}
  \begin{lstlisting}[caption=Definition of \textit{FunctionDeclaration} in SDF3 production notation]
FunctionDeclaration.Function = <
function <BindingIdentifier?><CallSignature> {
  <FunctionBody>
}
>
  \end{lstlisting}
  \caption{The translation of \textit{FunctionDeclaration} from both the ECMAScript and TypeScript language specifications to SDF3 production notation.}
  \label{fig:override-syntax}
\end{figure*}

Disambiguation based on context-free priorities are encoded in the production hierarchy of the ECMAScript language specification.
The primary target for disambiguation is \textit{PrimaryExpression}.
The production rule in SDF3 includes all \textit{PrimaryExpression}, while the ECMAScript defines a strict ordering of the production rules in the production hierarchy.
At the time of writing, no explicit \textit{context-free-priorities} have been defined in SDF3.
In the examples implemented in the language project, no ambiguation has been detected thus far.

\section{Type Checking}
\label{sec:type-checking}

The AST as defined by the syntax described in the previous section is traversed using NaBl \citep{Antwerpen:2016:CLS:2847538.2847543} to implement the type checking constraints as explained in section \ref{sec:structural-typing-typescript}.
NaBl is a constraint generation language, for which for every node in the AST a constraint can be generated.
As such, the typescript type checker consists of rules matching on various patterns of AST nodes to be able to generate the relevant constraints.

An example constraint for the \textit{IfStatement} defined in the previous section is shown in figure \ref{fig:if-constraints}.
These constraints are similar to the constraints defined in type checkers for programming languages such as (Mini-)Java and F\#.

\begin{figure*}
  \begin{lstlisting}[caption=Constraint generation for an \textit{If} AST node (representing an \textit{IfStatement})]
[[ If(condition, trueBlock) ^ (s) ]] :=
  [[ condition ^ (s) : valueTy ]],
  valueTy == BOOLEAN() | error $[Boolean expected, got [valueTy]],
  [[ trueBlock ^ (s) ]].
  \end{lstlisting}
  \begin{lstlisting}[caption=Constraint generation for an \textit{IfElse} AST node (representing an \textit{IfStatement})]
[[ IfElse(condition, trueBlock, falseBlock) ^ (s) ]] :=
  [[ condition ^ (s) : valueTy ]],
  valueTy == BOOLEAN() | error $[Boolean expected, got [valueTy]],
  [[ trueBlock ^ (s) ]],
  [[ falseBlock ^ (s) ]].
  \end{lstlisting}
  \caption{The constraints for the two forms of an \textit{IfStatement}.}
  \label{fig:if-constraints}
\end{figure*}

\subsection{Structural type checking}

In general, NaBl is well-suited for statements and expressions that are common in programming languages, such as the shown \textit{IfStatement}.
During this project, our research focussed to what extend does NaBl support structural type checking and, in the case that it does not, what is required to support it?
At the start of the project, the syntax definition included several language constructs that could be used to implement structural type checkings.
These constructs included: object declarations, variable assignments, interfaces, functions, function calls and primitives.
With these constructs several test cases and examples were developed that tested various aspects of structural type checking.
Most notably, the examples focused on function calls with object types while the test focused on variable declarations.

While NaBl does not assume any property of the programming language, structural type checking proved to be impossible at the time of writing.
The concrete problem was that at the moment of constraint generation, the run-time types were unknown.
In other words: for a variable assignment, it is not known which types are used in the type declaration and derived from the assigned value.
Normally, in NaBl the corresponding constraint is an equality or contravariance check (using respectively \textit{==} or \textit{<?}).
However, in a structural typed language, identity and sub-class relations are defined based on the current structure.
Thus, when generating constraints for a variable assignment, special checks need to be implemented if one of the arguments is an object.

\subsection{NaBl syntax proposal}

To support structural type checking, there is a syntax proposal for NaBl as shown in figure \ref{fig:nabl-syntax-proposal}.
The syntax proposal provides new syntax which is agnostic of the structural typing problem, but the new NaBl constructs can be used to implement structural type checking.

\begin{figure*}
  \begin{lstlisting}[caption=New constraint syntax constructs for NaBl]
Constraint.Exists  = <
  exists <VarId>@<Occurrence> in <Names> =\> <Constraint>>
Constraint.ForAll  = <
  forall <VarId>@<Occurrence> in <Names> =\> <Constraint>>
  \end{lstlisting}
  \begin{lstlisting}[caption=New custom relation definition syntax]
RelationPattern.RelationPattern = <
	(<RelationDefVariant>, <RelationDefVariant>) := <Constraint>.>

VarIds = {VarId ","}*
  \end{lstlisting}
  \begin{lstlisting}[caption=Example usage of the new syntax constructs]
relations
  sub: Type * Type {
    (List(one), List(other)) := true.
    ,
    (RECORD(one), RECORD(other)) :=
      forall d1@Field{x} in D(one)/Field => (
        d1 : ty1,
        exists d2@Field{x} in D(other)/Field => (
          d2 : ty2,
          ty1 <? ty2
        )
      ).
  }
}
  \end{lstlisting}
  \caption{The syntax proposal for NaBl using SDF3 templates.}
  \label{fig:nabl-syntax-proposal}
\end{figure*}

\section{Conclusion}

We have seen how structural types in TypeScript work as opposed to more common nominal type systems.

The definition of TypeScript and ECMAScript 2015 syntax with SDF3 is relatively straightforward
and even offers some improvements compared to the official ECMAScript 2015 specification.

Our attempt to model structural types with NaBL2 was met with more challenges.
Our efforts resulted in the insight that one pass of constraint generation and constraint solving
is not powerful enought to support structural and recursive types.

We have proposed the addition of custom relation definitions to allow language designers to define their own
subtype relations.
With this extension and two new constraints, \texttt{forall} and \texttt{exists}, language designers will 
be able to model more complex and diverse type systems.
As an example we have demonstrated that the proposed constructs can be used with Scope Graphs to model 
TypeScripts structural types, including unions and intersections.
\FloatBarrier

%% Bibliography
\bibliography{bibliography}

\end{document}
