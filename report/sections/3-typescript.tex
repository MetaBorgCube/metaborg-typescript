\section{Structural Typing in Typescript}
\label{sec:structural-typing-typescript}
TypeScript was built with structural typing as it's backbone, 
nearly any feature in the language can be described in terms of structual types.
The reason for this being that the goal of the language is to provide a type system
on top of the dynamically typed JavaScript language while not introducing any 
abstractions that have a conceptually high distance to the semantics of JavaScript.
The result of this effort means that compiled TypeScript code should be suitable 
for human consumption. Structural typing provides an excellent bridge from the 
dynamically typed JavaScript world to the statically typed world of TypeScript.
In this section we will introduce the most important features of TypeScript and
how they relate to structural types.
\bigskip
Structural types manifest themselves in TypeScript as \textit{Record Types}.
In the examples above, we have already encountered some record types, a set of name-type pairs.
Built directly on top of record types are some of the most important aspects of the language:
\begin{itemize}
\item classes
\item interfaces
\item union types
\item intersection types
\end{itemize}

Other notable features include: first class and higher order functions, parametric and ad-hoc polymorphism 
(generics) and several types for special values like \texttt{null} and \texttt{undefined}. It is 
worth noting that the subtype relation within TypeScript does not form a proper latice due to 
the \texttt{any} type. Namely, a value of type \texttt{any} is a subtype of every other type, 
while every type is simultaniously a subtype of any.
\bigskip
\subsection{Classes and Interfaces}
While syntax for defining classes in TypeScript is similar to Java, the structural typing makes classes semantically different form nominally typed languages.
The following snippet defines a simple class in TypeScript:

\begin{lstlisting}
class NonEmptyList<A> {
  head: A;
  tail: List<A>;
  constructor(head: A) {
    this.head = head;
    this.tail = new List();
  }
}
\end{lstlisting}

In any nominally typed language, it would not be possible to assign a $NonEmptyList$ to any variable that is not also a $NonEmptyList$.
In TypeScript this restriction does not apply. 
Any instance of a class can be assigned a variable with any type as long as the two are structurally in a subtype relationship.
This means that the following snippet is perfectly legal:

\begin{lstlisting}
class SignletonList<A> {
  head: A;
}

const s: SingletonList<number> = 
  new NonEmptyList(42);
\end{lstlisting}

Since names are irrelevant to the type checker, the following is also valid because the type of 
$s$ is structurally equivalent to $NonEmptyList<number>$.
\bigskip
\begin{lstlisting}
const s: { head: number, tail: List<number> } =
  new NonEmptyList(42);
\end{lstlisting}

This situation remains unchanged when we add 'methods' to the $NonEmptyList$ class:

\begin{lstlisting}
class NonEmptyList<A> {
  head: A;
  tail: List<A>;
  ...
  size(): number {
    return 1 + tail.size();
  }
}

const s: { 
  head: number, tail: List<number>,
  size: () => number  
} = new NonEmptyList(42);
\end{lstlisting}

From the above snippet we can see that classes are simply sugar for named object types.
The fields are directly translated to object members, and class methods are translated
to object members that have a function type corresponding to the method signature.
A consequence of this is that class methods are first class values.
Class constructors are translated to a special type of function outside the class scope that has the exact name of the class:

\begin{lstlisting}
class NonEmptyList<A> {
  ...
  constructor(head: A) {
    ...
  }
}
const a: typeof NonEmptyList = NonEmptyList;
const xs: NonEmptyList<number> = new a(42);
\end{lstlisting}

From this it appears that classes live in both the value namespace as well as the type namespace,
because an instance of $NonEmptyList$ ($xs$) can be created from a runtime value ($a$) on the last line.
In reality, $a$ represents the constructor for $NonEmptyList$ rather than the class itself.
We can therefore conclude that class constructors are first class values just like class methods.
The type of a constructor must different from the type of a function because the former can only be called with the $new$ keyword.
The $typeof$ keyword in the definition of $a$ is used in this case to signal that $a$ is a constructor.
Since constructors do not live inside the class scope, they have no impact type checking when handling instances of a class.
\bigskip
Inheritance is also handled in a straightforward manner.
When checking whether an instance of a class can be assigned to an object type, the compiler
simply takes all class members and all members of any parent class into account.
The rules that apply to classes can similarly be applied to interfaces, with the exception that interfaces do not have a constructor.

\subsection{Union and Intersection types}
Union and intersection types provide for a way to compose types, similar to sum and product types in many functional languages.
\begin{lstlisting}
interface X = { x: number }
interface Y = { y: string }
type U = X | Y
\end{lstlisting}

The last line in the code above defines $U$ to be the union of $X$ and $Y$. 
This means that $U$ can be either $X$ or $Y$ but not both, similar to discriminated unions in other languages.
Union types have the following subtype relationships:
\begin{itemize}
\item A union type $U$ is a subtype of a type $T$ if each type in $U$ is a subtype of $T$.
\item A type $T$ is a subtype of a union type $U$ if $T$ is a subtype of any type in $U$.
\end{itemize}
Similarly, union types have the following assignability relationships:
\begin{itemize}
\item A union type $U$ is assignable to a type $T$ if each type in $U$ is assignable to $T$.
\item A type $T$ is assignable to a union type $U$ if $T$ is assignable to any type in $U$.
\end{itemize}

When the types that make up the union have overlapping members, those members themselves become unions:
\begin{lstlisting}
interface X = { q: number }
interface Y = { q: string }
type U = X | Y
// U is equivalent to { q: number | string }
\end{lstlisting}
\bigskip
Intersection types describe values with multiple simultanious types. 
\begin{lstlisting}
interface X = { x: number }
interface Y = { y: string }
type I = X & Y
// I is equivalent to { x: number, y: string }
\end{lstlisting}

Just as with unions, when the types that make up an intersection have overlapping members, those members themselves become intersections:
\begin{lstlisting}
interface X = { q: number }
interface Y = { q: string }
type I = X & Y
// I is equivalent to { q: number & string }
\end{lstlisting}

Intersection types have the following subtype relationships:
\begin{itemize}
\item An intersection type $I$ is a subtype of a type $T$ if any type in $I$ is a subtype of $T$.
\item A type $T$ is a subtype of an intersection type $I$ if $T$ is a subtype of each type in $I$.
\end{itemize}
Similarly, intersection types have the following assignability relationships:
\begin{itemize}
\item An intersection type $I$ is assignable to a type $T$ if any type in $I$ is assignable to $T$.
\item A type $T$ is assignable to an intersection type $I$ if $T$ is assignable to each type in $I$.
\end{itemize}
\bigskip
Since functions are values in TypeScript, we can even define unions and intersections over functions.
The following example demonstrates the intersting case of an intersection over functions:
\begin{lstlisting}
type F1 = (a: string, b: string) => void;  
type F2 = (a: number, b: number) => void;

var f: F1 & F2 = (a: string | number, b: string | number) => { };  
f("hello", "world");
f(1, 2);
f(1, "test");
\end{lstlisting}

The last call to $f$ will be rejected by the compiler.
The type signature of $f$ makes it possible for the type checker to infer
that this function should only be called with two strings or two numbers.
If the type signature were removed, this program would be well-typed.
\bigskip
In the next sections we will discuss how these features would ideally be implemented in Spoofax.

