\section{Type Checking}
\label{sec:type-checking}

In this section we will dicuss our attempt at using NaBL2 \citep{Antwerpen:2016:CLS:2847538.2847543} to implement the type system as explained in section \ref{sec:structural-typing-typescript}.
NaBL2 is a constraint generation language specifically tailored towards the problem of name binding.
The language allows language designers to specify constraints over AST constructs, which are then solved in one pass by the runtime.
A general solution to the name binding problem is provided by a framework of scope graphs.
A scope graph consists of scopes, declarations and references to names in a program.
Constraints can be generated to construct such graphs, as well as to resolve paths from declarations to references in the graph.
Since the problem of type checking is very similar to that of name binding, NaBL2 can also be used to implement a type checker.

\subsection{Structural type checking}

In general, NaBL2 is well-suited for statements and expressions that are common in programming languages, 
such as the shown \textit{IfStatement}.
During this project, our research focussed on exploring to what extend NaBL2 supports structural type checking and, 
in the case that it does not, what would be required to support it.
Our syntax definition for TypeScript includes several language constructs that must participate in type checking.
These constructs included: object declarations, variable assignments, interfaces, functions, function calls and primitives.
Several test cases and examples were developed that tested various aspects of structural type checking with these constructs.

To demonstrate our attempt at implementing structural types, we will use object assignment as example.
Figure \ref{fig:running-example} shows a TypeScript program that should be rejected by the type checker 
because \texttt{Person} is a structural subtype of \texttt{Developer}. 
The assignment of \texttt{john} to a \texttt{Developer} on the last line is therefore illegal.
\begin{figure}[H]
\begin{lstlisting}
type Person = {
  name: string
}
type Developer = {
  name: string
  language: string
}
var john: Person = { name: 'John' }
var devJohn: Developer = john
\end{lstlisting}
\caption{Illegal object assignment in TypeScript.}
\label{fig:running-example}
\end{figure}

\subsection{Initial solution}
Object types can be modeled cleanly using scope graphs. 
The general approach is to define a type, \texttt{RECORD}, that wraps a \texttt{scope}.
The wrapped scope should contain the declatations of each field in the object type.
Checking whether two types are in a subtype relation then becomes the problem of checking whether
the the scope of the left-hand side \texttt{RECORD} type is a subset of the one on the right-hand side.
NaBL2 already supports scope subset checking.
This solution is shown in Figure \ref{fig:wrong-solution}.
The top constraint generation rule traverses over the \texttt{ObjectType} AST node.
This node corresponds to a type annotation on any variable or function argument in TypeScript.
The second generation rule traverses over variable declarations.
The second component of a \texttt{VariableDeclaration} is the declared type, 
while the third component denotes the value assigned to the variable.
Notice that \texttt{subseteq} is used to check that the left-hand side is a subset of the right-hand side.
Running this code on the example in Figure \ref{fig:running-example} produces the desired result.
NaBL2 will reject the last line based on the subset check.
\begin{figure*}[h]
\begin{lstlisting}
[[ ObjectType(fields) ^ (s): ty ]] :=
    new record_scope,
    ty == RECORD(record_scope),
    record_scope -P-> s,
    distinct/name D(record_scope)/Field,
    Map1 [[ fields ^ (record_scope) ]].

[[ VariableDeclaration(name, type@Id(_), value) ^ (s) ]] :=
    [[ type ^ (s) : ty ]],
    ty == RECORD(type_scope),
    [[ value ^ (s) : RECORD(value_scope) ]],
    D(type_scope)/Field subseteq/name D(value_scope)/Field,
    Value{name} <- s,
    Value{name} : ty !.
\end{lstlisting}
\caption{Naive solution to object type checking.}
\label{fig:wrong-solution}
\end{figure*}

While this solution is straightforward, there are two problems that prevent it from being useful.
First, a pattern match on \texttt{ type@Id(\_) } is needed to dertermine that the left-hand side is indeed an object type. 
This means that separate constraint generation rules are needed for checking non-object types, leading to code duplication.
Second, the \texttt{subseteq} operator only checks whether names of fields in both scopes are in a subset relation.
Types that are assigned to these fields are completely ignored.
This means that we could assign any object to a \texttt{Person} that has a \texttt{name} field, 
regardless of the type of that field.
The \texttt{subseteq} operator is insufficient to solve structural type checking.

\subsection{Extensible subtype relations}
The two problems illustrated above can be solved by making the NaBL2 subtype check user extensible.
Normally, checking compatibility of two types in NaBL2 is done by generating constraints for equality or subtype checks using \texttt{==} or \texttt{<?}.
In an ideal scenario, we would be able to use the \texttt{<?} operator to generate constraints for structural subtyping.
However, this would require the operator to perform different checks based on the types of its arguments.
For non-object types like strings and numbers, a simple non-recursive check is sufficient.
Structural types require a recursive check. 
As explained in section \ref{sec:structural-typing}, checking whether two object types in an equi-recursive system 
(like with TypeScript) are subtypes requires recursively unfolding both types and checking their equality in the fixed point.
Allowing users of NaBL2 to overload the \texttt{<?} operator is requires constraint generation and solving to be interleaved.
At solve-time, when a \texttt{<?} check is encountered, the solver must pause and evaluate the user-defined subtype relation, 
which in turn generates more constraints to be solved.
These constraints may include more, recursive, subtype checks in the case of recursive structural types like lists, trees, etc.
When custom subtype relations are allowed, 
the incorrect solution in Figure \ref{fig:wrong-solution} could be fixed by modifying it as shown in Figure \ref{fig:cool-solution}.
Notice that the pattern match is gone and the rule is completely polymorphic in the types of the left- and right-hand side.
\begin{figure*}
\begin{lstlisting}
[[ ObjectType(fields) ^ (s): ty ]] :=
    new record_scope,
    ty == RECORD(record_scope),
    record_scope -P-> s,
    distinct/name D(record_scope)/Field,
    Map1 [[ fields ^ (record_scope) ]].

[[ VariableDeclaration(name, type, value) ^ (s) ]] :=
    [[ type ^ (s) : ty ]],
    [[ value ^ (s) : valueTy ]],
    ty <? valueTy
    Value{name} <- s,
    Value{name} : ty!.
\end{lstlisting}
\caption{General solution when given the possibility of user-defined subtype checks.}
\label{fig:cool-solution}
\end{figure*}
In the next section, the syntactical details of defining custom subtype relations are discussed.

\subsection{NaBL2 syntax proposal}
To syntactically support the definition of custom subtype relations, 
we propose an extension to the SDF of NaBL2 as shown in Figure \ref{fig:nabl-syntax-proposal-relation}.
Each NaBL2 module can contain a \texttt{relations} section.
We propose adding a new relation pattern with two elements: 
\begin{itemize}
\item the left- and right-hand side of the subtype relation, denoted with common tuple-syntax
\item the constraint generated for these two patterns.
\end{itemize}
Note that this relation pattern supports one constraint, 
since multiple constraint can easilly be composed into one.
We also propose two new constraint construts: \texttt{forall} and \texttt{exists} as shown in Figure \ref{fig:nabl-syntax-proposal-constraint}.
These constructs can be used to check that a constraint holds over all declarations in a scope, 
or that there exists a declaration in a scope that satisfies a constraint.
These three new syntactical constructs can be used to define a subtype relation for the \texttt{RECORD} type,
illustrated in Figure \ref{fig:nabl-syntax-proposal-usage}.
The \texttt{forall} will generate the constraint after the \texttt{=>} symbol for each declaration in the \texttt{one} scope.
The \texttt{exists} construct ensures that \texttt{ty1 <? ty2} holds for at least one declatation in the \texttt{two} scope.
The \texttt{sub : Type * Type} annotation on the seconds line determines that the name of the relation is \texttt{sub},
which can be referenced in constraint generation rules as either \texttt{<sub?} or \texttt{<?}.
Different relations could be made for different purposes, and each relation can be overloaded for multiple types of arguments.
For example, as shown in the last lines of \ref{fig:nabl-syntax-proposal-usage}, one could define another subtype relation over \texttt{List} types.
Therefore, this syntax is not restricted to structural types and can also be useful when designing other kinds of type systems.

\begin{figure*}
\begin{lstlisting}
RelationPattern.RelationPattern = <
	(<RelationDefVariant>, <RelationDefVariant>) := <Constraint>.>

VarIds = {VarId ","}*
\end{lstlisting}
\caption{The syntax proposal for NaBL2 relation definitions.}
\label{fig:nabl-syntax-proposal-relation}
\end{figure*}

\begin{figure*}
\begin{lstlisting}
Constraint.Exists  = <
  exists <VarId>@<Occurrence> in <Names> =\> <Constraint>>
Constraint.ForAll  = <
  forall <VarId>@<Occurrence> in <Names> =\> <Constraint>>
\end{lstlisting}
\caption{New constraint syntax constructs for NaBL2.}
\label{fig:nabl-syntax-proposal-constraint}
\end{figure*}

\begin{figure*}
\begin{lstlisting}
relations
  sub: Type * Type {
    (RECORD(one), RECORD(other)) :=
      forall d1@Field{x} in D(one)/Field => (
        d1 : ty1,
        exists d2@Field{x} in D(other)/Field => (
          d2 : ty2,
          ty1 <? ty2
        )
      ).
    , (List(one), List(other)) := one <? two.
  }
}
\end{lstlisting}
\caption{Example usage of the new syntax constructs for structural types.}
\label{fig:nabl-syntax-proposal-usage}
\end{figure*}

\subsection{Union and Intersection types}
We will now demonstrate that with two additional, overloaded, versions of \texttt{forall} and \texttt{exists},
NaBL2 will be powerful enough to implement union and Intersection types as described in Section \ref{sec:structural-typing-typescript}.
Figure \ref{fig:union-impl} shows how this could be implemented.
Notice that we user \texttt{forall} and \texttt{exists} in a manner similar to object types, 
but rather operating on scopes, we now operate on lists of types.

\begin{figure*}
\begin{lstlisting}
relations
  sub: Type * Type {
    (UNION(List(types1)), UNION(List(types2))) :=
      forall t2 in types2 => (
        exists t1 in types1 => (ty1 <? ty2)
      ),
    (UNION(List(types)), ty) :=
      exists ty1 in types => (ty1 <? ty)
  }
}
\end{lstlisting}
\caption{Implementation of union types by using a custom subtype relation and overloaded \texttt{forall} and \texttt{exists} constraints.}
\label{fig:union-impl}
\end{figure*}

Intersection types are implemented with the same approach, but with the \texttt{forall} and \texttt{exists} constructs swapped to achieve behavior dual to unions.

\begin{figure*}
\begin{lstlisting}
relations
  sub: Type * Type {
    (INTERSECT(List(types1)), INTERSECT(List(types2))) :=
      forall t1 in types1 => (
        exists t2 in types2 => (ty1 <? ty2)
      ),
    (INTERSECT(List(types)), ty) :=
      forall ty1 in types => (ty1 <? ty)
  }
}
\end{lstlisting}
\caption{Intersection types, using the same constructs as with union types.}
\label{fig:intersect-impl}
\end{figure*}